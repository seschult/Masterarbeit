\FloatBarrier
\newpage
\section{An overview of modern HEP}
\subsection{Standard Model}
\label{SM}
According to the Standard Model, all the matter in the Universe consists of a limited set of elementary particles, which
interact with each other via different types of forces. 
The first group of those particles  are twelve elementary 
particles-fermions\footnote{
Corresponding to those particles, 
there is an associated antiparticle with the same mass 
and opposite electric charge.
%In this thesis the particle will refer to both
% -- particle and antiparticle unless it is  stated  explicitly.
}, 
which carry  spin $1/2$.
Those particles are six quarks and six leptons (see Tabs.~\ref{QUARKSTAB},~\ref{LEPTONSTAB}). 
The second group of elementary particles in the Standard 
\CQUARKSTAB
\CLEPTONSTAB
Model are the gauge bosons listed in  Tab.~\ref{BOSONSTAB}. 
Those particles are mediators of fundamental interactions.
Four of them -- $W^{+}$, $W^{-}$, $Z^{0}$ and
 $\gamma$ correspond to the electroweak interaction and one, the gluon, $g$
 -- to the strong interaction.
The hypothetical mediator that corresponds to gravity is called  a graviton.
The experimental discovery of a single graviton is not foreseen in the near future, as the energies
sufficient for its detection are too high for  modern accelerators and the 
cross section for the interaction of gravitons with matter is too low.
\CBOSONSTAB
Other particles, such as mesons and baryons, contain quarks and anti-quarks, 


Confinement is also reflected in the dependence of running the QCD coupling constant, which decreases with  momentum transferred
between interacting particles, $Q^2$:
\begin{equation*}
\alpha_{s}(Q^2)=\frac{12\pi}{(33-2n_f)ln(Q^2/\Lambda^2_{\rm{QCD}})}.
\end{equation*}

\begin{equation}
\alpha_{s}(Q^2)=\frac{12\pi}{(33-2n_f)ln(Q^2/\Lambda^2_{\rm{QCD}})}.
\label{hhh}
\end{equation}


Here $n_f$ stands for the number of involved flavours and $\Lambda_{\rm{QCD}}\approx200\mev$ is the
Figure example
\FHERAONE


 This property is called \textbf{asymptotic freedom}.

 and ChPT (Chiral perturbation theory)~\cite{Wise:1992hn} are used.



In Eq.~\ref{hhh} bala bala
