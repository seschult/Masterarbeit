%------------------------------------------------------------------------------
\chapter{The Standard Model of Partical Physics}
\label{sec:SM}
%------------------------------------------------------------------------------
The basic principal behinde the interactions, which are mediated by gauge Bosons, is  that the Lagrange density is invariant under local gauge transformation. More Precisse, the particals are discribed as Fields. The Time and Space evolution of those fields follows the Euler-Lagrange equation which can be determinde using Hamiltions variation Pricipall of minmal Action. 
Loclal gauge trenasoformation
The basic principal behinde the interactions, which are mediated by gauge Bosons, is  that the Lagrange density is invariant under local gauge transformation. More Precisse, the particals are discribed as Fields. The Time and Space evolution of those fields follows the Euler-Lagrange equation which can be determinde using Hamiltions variation Pricipall of minmal Action. 
Loclal gauge trenasoformation




\section{Fundamental  Particals, Fields and Forces }\label{key:SM 2}

The Standard Modell of Partical Physics \cite{glashow1961partial,glashow1970weak,gross1973asymptotically,politzer1973reliable,politzer1974asymptotic,salam1964electromagnetic,weinberg1967model}, is a re-normalizable and relativistic Quantum Field Theory (QFT) of the fundamental fermions (spin $s =1/2$) and their interactions mediated by gauge bosons (spin $s = 1$).  It is based on a small number of particles and symmetry principles. Basically, as shown in \cref{fig:SM}, there are three fermion generations. Each of them consists of  a up- and a down type quark, a charged lepton and a corresponding neutrino. \\ 

It's basic structure is depicted in   , with three generations of fermions (quarks and leptons), of which each conists of a up- and a down type quark, a charged lepton and a corresponding neutrino.  The existence of further heavy  generation of up- and down-type has been exluded as for Mt and Mb wit 95 CL.
Additionally the mass of a possible 4th lepton generation is limmited by the mass of the Z boson, and its lifetime Measuremntes confirms only three light neutrion generatis. \\ 





In the theory the particles are described as fields $\psi$, which follow 

\noindent The basic principal behinde the interactions, which are mediated by gauge Bosons, is  that the Lagrange density is invariant under local gauge transformation. More Precisse, the particals are discribed as Fields. The Time and Space evolution of those fields follows the Euler-Lagrange equation which can be determinde using Hamiltions variation Pricipall of minmal Action. 
Loclal gauge trenasoformation
%based on a small number of fields and symetrieprinciels  



Local gauge invariance requires the Standard Model Lagrangian to remain invariant when transforming the fermion fields in the following manner:
ψ → ψ exp(−iα j(x) f j). (2.1)
This principle is only valid in the presence of (massless) fields which transform in the same manner as the fermion fields thus compensating the fermion field transformations in each point of space and time. Based on the choice of the local gauge symmetry group the strong and electroweak interaction theories emerge - introducing a set of bosonic particles intermediating the corresponding forces. Following Noether’s theorem, each symmetry group introduces field-charges which are conserved under local gauge transfromation.




% Standard model of physics
% Author: Carsten Burgard

%%%<


\begin{figure}
	\centering



\newcommand\particle[7][white]{%
	\begin{tikzpicture}[x=1cm, y=1cm]
	\path[fill=#1,] (0.1,0) -- (0.9,0)
	arc (90:0:1mm) -- (1.0,-0.9) arc (0:-90:1mm) -- (0.1,-1.0)
	arc (-90:-180:1mm) -- (0,-0.1) arc(180:90:1mm) -- cycle;
	\ifstrempty{#7}{}{\path[]
		(0.6,0) --(0.7,0) -- (1.0,-0.3) -- (1.0,-0.4);}
	\ifstrempty{#6}{}{\path[] (0.7,0) -- (0.9,0)
		arc (90:0:1mm) -- (1.0,-0.3);}
	\ifstrempty{#5}{}{\path[] (1.0,-0.7) -- (1.0,-0.9)
		arc (0:-90:1mm) -- (0.7,-1.0);}
	\draw[\ifstrempty{#2}{dashed}{black}] (0.1,0) -- (0.9,0)
	arc (90:0:1mm) -- (1.0,-0.9) arc (0:-90:1mm) -- (0.1,-1.0)
	arc (-90:-180:1mm) -- (0,-0.1) arc(180:90:1mm) -- cycle;
	\ifstrempty{#7}{}{\node at(0.825,-0.175) [rotate=-45,scale=0.2] {#7};}
	\ifstrempty{#6}{}{\node at(0.9,-0.1)  [nosep,scale=0.17] {#6};}
	\ifstrempty{#5}{}{\node at(0.9,-0.9)  [nosep,scale=0.2] {#5};}
	\ifstrempty{#4}{}{\node at(0.1,-0.1)  [nosep,anchor=west,scale=0.25]{#4};}
	\ifstrempty{#3}{}{\node at(0.1,-0.85) [nosep,anchor=west,scale=0.3] {#3};}
	\ifstrempty{#2}{}{\node at(0.1,-0.5)  [nosep,anchor=west,scale=1.5] {#2};}
	\end{tikzpicture}
}


	\begin{tikzpicture}[x=1.2cm, y=1.5cm]
	%\draw (-0.5,0.5) rectangle (4.4, -1.5);
	%\draw (-0.6,0.6) rectangle (5.0,-2.5);
	%\draw (-0.7,0.7) rectangle (5.6,-3.5);
	
	\node at(0, 0)   {\particle[green!20!white]
		{$u$}        {up}       {$2.3$ MeV}{1/2}{$2/3$}{R/G/B}};
	\node at(0,-1)   {\particle[green!20!white]
		{$d$}        {down}    {$4.8$ MeV}{1/2}{$-1/3$}{R/G/B}};
	\node at(0,-2)   {\particle[blue!20!white]
		{$e$}        {electron}       {$511$ keV}{1/2}{$-1$}{}};
	\node at(0,-3)   {\particle[blue!20!white]
		{$\nu_e$}    {$e$ neutrino}         {$<2$ eV}{1/2}{}{}};
	\node at(1, 0)   {\particle[green!20!white]
		{$c$}        {charm}   {$1.28$ GeV}{1/2}{$2/3$}{R/G/B}};
	\node at(1,-1)   {\particle[green!20!white] 
		{$s$}        {strange}  {$95$ MeV}{1/2}{$-1/3$}{R/G/B}};
	\node at(1,-2)   {\particle[blue!20!white]
		{$\mu$}      {muon}         {$105.7$ MeV}{1/2}{$-1$}{}};
	\node at(1,-3)   {\particle[blue!20!white]
		{$\nu_\mu$}  {$\mu$ neutrino}    {$<190$ keV}{1/2}{}{}};
	\node at(2, 0)   {\particle[green!20!white]
		{$t$}        {top}    {$173.2$ GeV}{1/2}{$2/3$}{R/G/B}};
	\node at(2,-1)   {\particle[green!20!white]
		{$b$}        {bottom}  {$4.7$ GeV}{1/2}{$-1/3$}{R/G/B}};
	\node at(2,-2)   {\particle[blue!20!white]
		{$\tau$}     {tau}          {$1.777$ GeV}{1/2}{$-1$}{}};
	\node at(2,-3)   {\particle[blue!20!white]
		{$\nu_\tau$} {$\tau$ neutrino}  {$<18.2$ MeV}{1/2}{}{}};
	\node at(3,-3)   {\particle[red!50!white]
		{$W^{\hspace{-.3ex}\scalebox{.5}{$\pm$}}$}
		{}              {$80.4$ GeV}{1}{$\pm1$}{}};
	\node at(4,-3)   {\particle[red!50!white]
		{$Z$}        {}                    {$91.2$ GeV}{1}{}{}};
	\node at(3.5,-2) {\particle[red!50!white!]
		{$\gamma$}   {photon}                        {}{1}{}{}};
	\node at(3.5,-1) {\particle[red!50!white]
		{$g$}        {gluon}                    {}{1}{}{color}};
	\node at(5,0)    {\particle[gray!50!white]
		{$H$}        {Higgs}              {$125.1$ GeV}{0}{}{}};
%	\node at(6.1,-3) {\particle
%		{}           {graviton}                       {}{}{}{}};
	
	%\node at(4.25,-0.5) [force]      {strong force (color)};
	%\node at(4.85,-1.5) [force]    {electromagnetic force (charge)};
	%\node at(5.45,-2.4) [force] {weak nuclear force (weak isospin)};
	%\node at(6.75,-2.5) [force]        {gravitational force (mass)};
	
	\draw [<-] (2.5,0.3)   -- (2.7,0.3)          node [legend] {charge};
	\draw [<-] (2.5,0.15)  -- (2.7,0.15)         node [legend] {colors};
	\draw [<-] (2.05,0.25) -- (2.3,0) -- (2.7,0) node [legend]   {mass};
	\draw [<-] (2.5,-0.3)  -- (2.7,-0.3)         node [legend]   {spin};
	
	\draw [mbrace] (-0.8,0.5)  -- (-0.8,-1.5)
	node[leftlabel] {quarks};
	\draw [mbrace] (-0.8,-1.5) -- (-0.8,-3.5)
	node[leftlabel] { leptons};
	\draw [mbrace] (-0.5,-3.6) -- (2.5,-3.6)
	node[bottomlabel]
	{ fermions\\};
	\draw [mbrace] (2.5,-3.6) -- (4.5,-3.6)
	node[bottomlabel] {gauge bosons\\};
	
%	\draw [brace] (-0.5,.8) -- (0.5,.8) node[toplabel]         {stable };
	%\draw [brace] (0.5,.8)  -- (2.5,.8) node[toplabel]         {unstable };
	%\draw [brace] (2.5,.8)  -- (4.5,.8) node[toplabel]          {force carriers};
	\draw [brace] (4.5,.8)  -- (5.5,.8) node[toplabel]       {goldstone\\bosons };
	%\draw [brace] (5.5,.8)  -- (7,.8)   node[toplabel] {outside\\standard %model};
	
	\node at (0,.8)   [generation] {1\tiny st};
	\node at (1,.8)   [generation] {2\tiny nd};
	\node at (2,.8)   [generation] {3\tiny rd};
	\node at (2.8,.8) [generation] {\tiny generation};
	
	
	\end{tikzpicture}

	\caption{Graphical } \label{fig:SM}
\end{figure}


  
%%% Local Variables: 
%%% mode: latex
%%% TeX-master: "../mythesis"
%%% End: 
\clearpage
\begin{figure}
\centering
\includegraphics[width=0.7\linewidth, height=0.5\textheight]{"Pics/Stupid People everywhere"}
\caption{}
\label{fig:StupidPeopleeverywhere}
\end{figure}

\clearpage


\section{QCD}
\begin{figure}
\centering
\includegraphics[width=0.7\linewidth, height=0.5\textheight]{"Pics/Stupid People everywhere"}
\caption{}
\label{fig:StupidPeopleeverywhere}
\end{figure}


\clearpage
\section{Elektroweak inteaction theory and Higgs Mechanism}
\begin{figure}
\centering
\includegraphics[width=0.7\linewidth, height=0.5\textheight]{"Pics/Stupid People everywhere"}
\caption{}
\label{fig:StupidPeopleeverywhere}
\end{figure}


\clearpage