%%\documentclass[hyperref={colorlinks=true},10pt]{beamer}
\documentclass[green,compress,10pt]{beamer}
\usepackage[latin1]{inputenc}
\usepackage{textcomp}
\usepackage{beamerthemesplit,graphics,color,graphicx,amsmath,amssymb,mathrsfs}
%\usepackage{appendixnumberbeamer}
\usepackage{xcolor,subfigure}
\usepackage{url}
\usepackage{multimedia}
\usepackage{rotating}
\usepackage{colortbl}
\usepackage{pifont}
\usepackage[absolute,overlay]{textpos}
\usepackage{path}
%\usepackage{feynmp}
\usepackage{underscore}
\newcommand{\ttbar}{$t\bar{t}$}
\newcommand{\ttbargamma}{$t\bar{t}\gamma\,$}
\newcommand{\invfb}{fb$^{-1}$}
\newcommand{\mtop}{m$_{\rm{top}}$}
\newcommand{\etmiss}{E$_{\rm{T}}^{\rm{miss}}$}
\newcommand{\pT}{p$_{\text{T}}$}
\newcommand{\vr}{V_{\rm{R}}}
\newcommand{\vl}{V_{\rm{L}}}
\newcommand{\gr}{g_{\rm{R}}}
\newcommand{\gl}{g_{\rm{L}}}
\newcommand{\fr}{F_{\rm R}}
\newcommand{\fl}{F_{\rm L}}
\newcommand{\fz}{F_{\rm 0}}

\newcommand{\fR}{$F_{\rm R} \;$}
\newcommand{\fL}{$F_{\rm L} \;$}
\newcommand{\fZ}{$F_{\rm 0} \;$}
\newcommand{\nz}{N_{\rm 0} \;}
\newcommand{\nl}{N_{\rm L} \;}
\newcommand{\nr}{N_{\rm R} \;}
\newcommand{\Nz}{$N_{\rm 0} \;$}
\newcommand{\Nl}{$N_{\rm L} \;$}
\newcommand{\Nr}{$N_{\rm R} \;$}

\definecolor{brightgray}{rgb}{0.95,0.95,0.95}

\mode<presentation>
{
 \setbeamertemplate{background canvas}[vertical shading][bottom=white!10,top=white!10]
 \usetheme{Ilmenau}
 \usecolortheme{riceowl}
 \usefonttheme[onlysmall]{structurebold}

 
 \setbeamertemplate{footline}{                                                                                    
                                                  
   \begin{beamercolorbox}[wd=\paperwidth,ht=2.55ex,dp=1ex,dp=1ex]{date in head/foot}                                                          
                  
 \hspace{1ex} \scriptsize{\insertframenumber / \insertpresentationendpage \hspace{2.99cm}  -- Top quark mass @ 13TeV --  \hspace{2.01cm} Sebastian Schulte }     
% \hspace{1ex} \scriptsize{\insertframenumber / 3 \hspace{3.85cm}  -- Status report: reprocessing --  \hspace{2.61cm} Andrea Knue}

 \end{beamercolorbox}}

 %    \end{beamercolorbox}}%                                        
 
}


\setbeamercolor{button}{bg=magenta,fg=black}

\definecolor{hellgrau}{rgb}{0.95,0.95,0.95}
\newcommand{\Tipp}[1]{\textcolor{blue}{Tipp:} #1}
\newcommand{\heading}[1]{\textcolor{blue}{#1}}
\newcommand{\kommentar}[1]{}
\beamertemplatenavigationsymbolsempty
\setbeamercovered{dynamic}

\begin{document}
\title[]{Measurement of the Top Quark Mass in 
	the\\ $t\={t}\rightarrow$ lepton+jets channel \\form $\sqrt{s}=13$TeV ATLAS data}
\vspace{2cm}

\author{{{\bf Sebastian~Schulte}, Andrea~Knue, StefanKluth, Richard Nisius}}
%\date{Fourth Annual LHCP Conference \\ -- Lund, 13th-18th June 2016 --}

\frame{
\begin{textblock}{14}(0.75, 1.75)
\titlepage
\end{textblock}
\vspace{-1.05cm}
\begin{textblock}{6}(-0.55, 9.0)
\begin{figure}
\centering
\subfigure{\includegraphics[width=0.9in]{Logo/Tum.jpg}}
\end{figure}
\end{textblock}

\begin{textblock}{7.5}(4.0, 8.9)
	\begin{figure}
		\centering
		\subfigure{\includegraphics[width=1.85in]{Logo/MPPGr.png}}
	\end{figure}
\end{textblock}

\begin{textblock}{6}(10.5, 9.0)
\begin{figure}
\centering
\subfigure{\includegraphics[width=0.9in]{Logo/ATLAS.png}}
\end{figure}
\end{textblock}
}


\section{Introduction}


\frame{
\frametitle{Why Measuring the Top-Quark mass?}

}

\frame{
	\frametitle{How the Data is taken?}
	
}
\frame{
	\frametitle{How is the Top-Qark mass measured?}
	\begin{textblock}{18}(0.5, 2.8)	
	\bfseries{Measurement is based on a 3D-Template method:}
	\end{textblock}
\begin{textblock}{18}(0.5, 4.0)	
	\begin{itemize}
		\item Variable 1: $m_{top}^{reco}$ from reconstructed Events 
		\item Variable 2: $m_{W}^{reco}$ from chosen jet permutation, sensitive to JSF
		\item Variable 2: $R_{bq}^{reco}$ from chosen jet permutation, sensitive to bJSF
	
  \end{itemize}
	\end{textblock}
	
	
	\begin{textblock}{5}(2.5, 7.5)		
		
	
	
		$R_{bq} ^{reco,1b}= \frac{  p_{T}^{b_{tag}}    }{(p_{T}^{W_{jet1}} + p_{T}^{W_{jet2}})/2  } $
	
	\end{textblock}
	
		\begin{textblock}{5}(8.5, 7.5)		
			
			$R_{bq} ^{reco,2b}= \frac{  p_{T}^{b_{had}} + p_{T}^{b_{lep}}     }{p_{T}^{W_{jet1}} + p_{T}^{W_{jet2}}  } $
		\end{textblock}
		
	
	
	
	
	
	
	\begin{textblock}{9}(3.5, 9.5)	
		\begin{block}{Determination of $m_{top}:$}
			\begin{itemize}
				\item Need fully reconstruction of $t\={t}$-finale state
				\item Template parametrisation of the 3 variables
				\item Unbinned likelihood fit is performed
			\end{itemize}
		
		\end{block}
		\end{textblock}	
	}
	




\section{Event selection}
%%%%%%%%%%%%%%%%%%%%%%%%%%%%%%%%%%%%%%%%%%%%%%%%%%%%%%%
\frame{
	\frametitle{\bfseries{Object definition for 2016 data}}
	
		\begin{textblock}{7}(0.5, 2.5)	
			\begin{block}{\bfseries{Electrons}}
				\begin{itemize}
					\item $E_T > $28 GeV, $\mid \eta  \mid< $2.47
					\item Gradient isolation, TightLH
					\item HLT_e26_lhtight_nod0_ivarloos,
					HLT_e60_lhmedium_nod0, 
					HLT_e140_lhloose_nod0
				\end{itemize}
				
			\end{block}	
	
			\end{textblock}	
	
		\begin{textblock}{7}(0.5, 8.5)	
		\begin{block}{\bfseries{Small-R jets}}
			\begin{itemize}
				\item antiKt R = 0.4, EM-Jets
				\item JVT $>$0.59 for $p_T <$ 60GeV and $\mid \eta  \mid< $2.4
				\item b-tagging: MV2_c10, 77\% WP
			\end{itemize}
			
		\end{block}	
		
	\end{textblock}	
	
	
	
	
			\begin{textblock}{7}(8.5, 2.5)	
		\begin{block}{\bfseries{Muons}}
			\begin{itemize}
				\item $E_T > $28 GeV, $\eta < $2.47
				\item Medium, Gradient isolation
				\item HLT_mu26_ivarmedium,
				 HLT_mu50
				 \space
				 \space
			\end{itemize}
			
		\end{block}	
		
	
	\end{textblock}	
	
		\begin{textblock}{7}(8.5, 8.5)	
		\begin{block}{\bfseries{MET/MTW}}
			\begin{itemize}
			\item $E_T^{miss} >$ 20GeV
			\item $E_T^{miss} + m_T^{W}>$ 60GeV
			\end{itemize}
			
		\end{block}	
		
	\end{textblock}	
	
		\begin{textblock}{17}(1.5,14.5)	
			\bfseries{AnalysisTop-02-04-27}, with 25 fb-1 for 2016 data
			\href{https://twiki.cern.ch/twiki/bin/viewauth/AtlasProtected/TopMassNtuplesRun2}{\beamergotobutton{Top Mass Ntuple production}}
    	\end{textblock}
	
}
%%%%%%%%%%%%%%%%%%%%%%%%%%%%%%%%%%%%%%%%%%%%%%%%%%%%%%%


\frame{
	\frametitle{Pre-selection}
		
	\begin{itemize}
	
			
	
		\item At least one good primary vertex with five associated tracks
		\vspace{0.2cm}
		\item Exactly one isolated high $ p_T$ lepton
			\vspace{0.2cm}
	\item At least 4 central jets with high$ p_T$ 
		\vspace{0.2cm}
	\item 1 or 2 b-tagged jets 
			\vspace{0.2cm}
	\item Cuts on   $E^{miss}_T$ ,  $m_T^W$  or  $E^{miss}_T$  + $m_T^W$             
			\vspace{0.2cm}
		\item W+jets normalization and HF fraction estimated from data
			\vspace{0.2cm}
	\item	Multijet background obtained from data in control region
	

	\end{itemize}
	


	
}

\section{Data/MC agreement 13 TeV}

\frame{
	\frametitle{Event yields after preselection}
	
	\begin{textblock}{18}(1.5, 2.5)	
		
\end{textblock}	
	
\begin{textblock}{18}(1.5, 9.5)	
	\begin{itemize}
		\item Background contamination dominated by W + Jets
		\vspace{0.2cm}
		\item Mass dependence of single-top $\Rightarrow $ include in signal
		\vspace{0.2cm}
		\item Reduction of background via cuts on 2 b-tagged jets
		\vspace{0.2}
		\item \bfseries{Good data/MC agreement }
		 
	\end{itemize}
	
	
	
\end{textblock}
	
	
}


\frame{
	\frametitle{Data/MC agreement}
	
	\begin{textblock}{18}(0.5,10.5)	
		\begin{itemize}
	\item	good Data/MC agreement for jet multiplicities and Missing $E_T$ 
		\vspace{0.2cm}
	\item	see larger discrepancies for higher  b-tag multiplicities
		\vspace{0.2cm}
		\item when we cut on at least two b-tagged jets, the Data/MC agreement 
		gets worse
		\end{itemize}
		
		
		
	\end{textblock}
}




\frame{
	\frametitle{Data/MC agreement}
		
		
		\begin{textblock}{18}(0.5,10.5)	
		\begin{itemize}
         \item Multijet estimate: preliminary estimate from 2015 data from Nedaa 
        scaled to 2016 luminosity 
		\vspace{0.2cm}
		\item Therefore see discrepancy at low electron $p_T$
		
	\end{itemize}

\end{textblock}
}
\frame{
	\frametitle{Data/MC agreement}
	\begin{textblock}{18}(0.5,10.5)
	\begin{itemize}
		\item Used four jets for KLFitter: priority for b-tagged jets
	\vspace{0.2}
\item	Do not use b-tag veto mode, but working point
	
	

	

\end{itemize}
\end{textblock}
}


\frame{
	\frametitle{Data/MC agreement}
	\begin{textblock}{18}(0.5,10.5)
		\begin{itemize}
		
			
			\item Variables used for likelihood fit
			\vspace{0.2cm}
			\item	Did not store information to calculate MW_reco, will be shown in the next presentation
			
			
		\end{itemize}
	\end{textblock}
}


\section{Event reconstruction}

\frame{
	\frametitle{$t\={t}$-final state}

	\begin{textblock}{10}(0.5,2.5)
%	\begin{alertblock}
	\begin{itemize}
		
	
	\item 4 jet event $\Rightarrow$ 24 possible jet-parton assignments
	\vspace{0.2cm}
	\item 12 permutations left since light jets from W are  indistinguishable
	\vspace{0.2cm}
	\item Kinematic liklihood fit with KLFitter\\
	\vspace{0.2cm}
	$\Rightarrow$ Chose best  permutations for calculation 


	\end{itemize}
%\end{alertblock}
    \end{textblock}
}

\frame{
	\frametitle{Reconstruction with KLFitter}
	\begin{textblock}{10}(0.5,2.5)
	\begin{itemize}
		\item KLFitter input: charged lepton, missing $E_T$ and at least four jets\\
		\vspace{0.2cm}
		$\Rightarrow$ one or two b-tagged jets + untagged jets with highest $p_T$
		\vspace{0.2cm}
		\item Definition of kinematic Likelihood:\\
		\begin{itemize}
			\item Transfer functions ($W$) for Detector response
			\item{Breit-Wigner distributions ($BW$)}
			\item different options to use b-tagging information 
		\end{itemize}
		
		
	\end{itemize}
	
	
	
	\end{textblock}
}

\section{Template Parametrisation} 

\frame{
	\frametitle{Template parametrisation}
	
	
	\begin{textblock}{10}(0.5,2.5)
		%	\begin{alertblock}
		\begin{itemize}
			
			
			\item Schau in die Int note!!!! 
			
			
		\end{itemize}
		%\end{alertblock}
	\end{textblock}
}
	
	
	
}
\frame{
	\frametitle{Workflow for Optimization studies}
	
}

\frame{
	\frametitle{In the following}
	
}
\frame{
	\frametitle{Previous results}
	
}
	
	\frame{
		\frametitle{Summery Conclusion}
		
	}
	
	



\end{document}
